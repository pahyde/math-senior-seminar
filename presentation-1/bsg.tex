\documentclass{beamer}
\usepackage{amsmath}
\usepackage{amsthm}
\usepackage{graphicx}
\graphicspath{ {./images/}  }

%Information to be included in the title page:
\title{Review: Build a Sporadic Group in Your Basement}
\author{Parker Hyde}
\institute{Georgia State University}
\date{\today}

\begin{document}

\frame{\titlepage}

\begin{frame}
    \frametitle{What is a Group?}
    \begin{itemize}
        \item<2-> \( (G, *) \)
        \item<3->satisfying the properties:
            \begin{enumerate}
                \item closure \( a, b \in G \implies a * b \in G \)
                \item associativity: \(  a * (b * c) = (a * b) * c \ \ \forall a,b,c \in G \) 
                \item existence of identity: \( \exists e \in G \text{ s.t. } a*e = e*a = a \ \forall a \in G \) 
                \item existence of inverses: \( \forall a \in G,  \exists a^{-1} \text { s.t. } a * a^{-1} = a^{-1} * a = e\) 
            \end{enumerate}
    \end{itemize}
    \only<4->{
        \bigbreak
        Examples:
        \( (\mathbb{Z}, +) \) \ 
        %\only<5->{\( S_{n} = (\sigma_n, \circ)\)}
        %\bigbreak
        %\only<6->{ \( \begin{pmatrix}
        %        1 & 2 & 3 & \cdots & n-1 & n \\
        %        2 & 3 & 4 & \cdots &  n  & 1
        %      \end{pmatrix} \in \sigma_n
        %  \) }
    }
\end{frame}
\newtheorem{exmp}{Example}

\begin{frame}
    \frametitle{What is the Sporadic group \( M_{24} \)?}
    \begin{itemize}
        \item<2-> \( M_{24} \le S_{24} = (\pi_{24}, \circ) \)
            \only<3->{\( \ni \begin{pmatrix}
                1 & 2 & 3 & \cdots & 23 & 24 \\
                2 & 3 & 4 & \cdots & 24  & 1
        \end{pmatrix} \)}
    \item<4-> \( |M_{24}| = 2^{10} \cdot 3^3 \cdot 5 \cdot 7 \cdot 11 \cdot 23 = 244823040 \)
        \bigbreak
    \item<5-> A Sporadic Group is a \textit{special} of finite simple group.
    \item<6-> A simple group is a group with no nontrivial normal groups.
    \item<7-> 
        \begin{example}
        \end{example}
        \( 5\mathbb{Z} = \{..., -10, -5, 0, 5, 10, ...\} \unlhd \mathbb{Z} \)
    \item<8-> \( \implies \frac{\mathbb{Z}}{5\mathbb{Z}} 
            = \mathbb{Z}_5
            = \{ \overline{0}, \overline{1}, \overline{2}, \overline{3}, \overline{4} \}\)
            \bigbreak
        \item<9-> \( M_{24} \) is an unusual finite simple group.
    \end{itemize}
\end{frame}

\begin{frame}
    \frametitle{Mathieu's Construction}
    \begin{itemize}
        \item<1->\( M_{24} \) was originally constructed by the following three arbitrary
            permutations.
            \footnotesize{
            \begin{align*}
                a &= (1,2,3,...,23) \\
                b &= (3,17,10,7,9)(5,4,13,14,19)(11,12,23,8,18)(21,16,15,20,22) \\
                c &= (1,24)(2,23)(3,12)(4,16)(5,18)...(9,21)(11,17)(13,22)(19,15)
            \end{align*}
        }%
    \item<2-> link: http://www.netlify/app.sdfjewhwef.com
    \end{itemize}
\end{frame}

\begin{frame}
    \frametitle{The Extended Golay Code}
    \begin{itemize}
        \item<2-> First, write down the numbers \(0, \ 1 , \ 2, \ ..., \ 2^{24} - 1\).
        \item<3-> Consider their binary representation as \( 24 \)-bit words.
        \item<4-> Add \( 0 \) to the list.
        \item<5-> Add any number differing in at \( 8 \) bit positions from previously 
            added words.
        \item<6-> Which gives an extended Golay Code:
        \item<4->
            \begin{align*}
                000000000000000000000000 \\
                \only<5->{ 000000000000000011111111 \\}
                \only<6->{  000000000000111100001111 \\
                000000000011001100110011 \\
                000000000101010101010101\\
                ...........................................\\
                ...........................................
            }
            \end{align*}
    \end{itemize}
\end{frame}

\begin{frame}
    \frametitle{Two Extended Golay Code Models}
    \begin{columns}
        \column{0.5\textwidth}
        Quadratic Residue (R)
        \setcounter{MaxMatrixCols}{25}                                   
        \setlength\arraycolsep{0.5pt}                                    
        
        \scriptsize{
        \[                                                               
            \begin{bmatrix}                                          
                1&1&1&1&0&1&0&1&1&0&0&1&1&0&0&1&0&1&0&0&0&0&0&1 \\    
                0&1&1&1&1&0&1&0&1&1&0&0&1&1&0&0&1&0&1&0&0&0&0&1 \\    
                0&0&1&1&1&1&0&1&0&1&1&0&0&1&1&0&0&1&0&1&0&0&0&1 \\    
                0&0&0&1&1&1&1&0&1&0&1&1&0&0&1&1&0&0&1&0&1&0&0&1 \\    
                0&0&0&0&1&1&1&1&0&1&0&1&1&0&0&1&1&0&0&1&0&1&0&1 \\    
                0&0&0&0&0&1&1&1&1&0&1&0&1&1&0&0&1&1&0&0&1&0&1&1 \\    
                1&0&0&0&0&0&1&1&1&1&0&1&0&1&1&0&0&1&1&0&0&1&0&1 \\    
                0&1&0&0&0&0&0&1&1&1&1&0&1&0&1&1&0&0&1&1&0&0&1&1 \\    
                1&0&1&0&0&0&0&0&1&1&1&1&0&1&0&1&1&0&0&1&1&0&0&1 \\    
                0&1&0&1&0&0&0&0&0&1&1&1&1&0&1&0&1&1&0&0&1&1&0&1 \\    
                0&0&1&0&1&0&0&0&0&0&1&1&1&1&0&1&0&1&1&0&0&1&1&1 \\    
                1&0&0&1&0&1&0&0&0&0&0&1&1&1&1&0&1&0&1&1&0&0&1&1\\     
                    \end{bmatrix}                                                
        \]
        }
        \column{0.5\textwidth}
        Block-Substitution (B)
        \bigbreak 
        \setcounter{MaxMatrixCols}{25}                                   
        \setlength\arraycolsep{2pt}                                    
        \scriptsize{
            \[
                I = \begin{bmatrix}
                        1 & 0 & 0 \\
                        0 & 1 & 0 \\
                        0 & 0 & 1 
                    \end{bmatrix}
                    \mkern9mu
                \overline{I} = \begin{bmatrix}
                        0 & 1 & 1 \\
                        1 & 0 & 1 \\
                        1 & 1 & 0 
                    \end{bmatrix}
                    \mkern9mu
                J = \begin{bmatrix}
                        0 & 1 & 1 \\
                        1 & 0 & 1 \\
                        1 & 1 & 0 
                    \end{bmatrix}
            \] 
             \setlength\arraycolsep{5pt}                                    
            \[
               B = \begin{bmatrix}                             
                    I & 0 & 0 & 0 & \overline{I} & I & I & J\\  
                    0 & I & 0 & 0 & J & \overline{I} & I & I\\  
                    0 & 0 & I & 0 & I & J & \overline{I} & I\\  
                    0 & 0 & 0 & I & I & I & J & \overline{I}    
                \end{bmatrix}                                   
            \] 
           \bigbreak 
           \bigbreak 
           \bigbreak 
        }
    \end{columns}
    \only<2->{
    \[
        \sigma = (1,2,3,4,5,6,7,8,9,10,11,12,13,14 ..., 22, 23)(24) 
    \]
}
    \only<3-> {
    \[
        \rho = (1,2,3)(4,5,6)(7,8,9)(10,11,12)...(22,23,24) 
    \]
}
\end{frame}

\end{document}
